\documentclass[11pt, a4paper]{report}
\usepackage[margin=3cm]{geometry}
%\usepackage[margin=4cm]{geometry}

\usepackage{verbatim} 
\usepackage{graphicx}
%\usepackage{subfigure}

% float for setting placement prefs
\usepackage{float}
\usepackage{amsmath}
\usepackage{algorithm,algpseudocode}
\usepackage{caption}
\usepackage{color}

% floatstyle to place all figs in box
%\floatstyle{boxed} 
%\restylefloat{figure}

\usepackage{setspace}
%\singlespacing
%\onehalfspacing
%\doublespacing
%\setstretch{1.1}

\usepackage{url}
% By default the URLs are put in typewriter type in the body and the
% bibliography of the document when using the \url command.  If you are
% using many long URLs you may want to uncommennt the next line so they
% are typeset a little smaller.
\renewcommand{\UrlFont}{\small\tt}

% amsthm to include labelled enunciations
%\begin{mydef}
%Here is a new definition
%\end{mydef}
\usepackage{amsthm}
\usepackage{fancyhdr}
\usepackage{subfig}

%\usepackage{hyperref} %TODO no borders!
%\hypersetup{plainpages=false}

\newcommand{\HRule}{\rule{\linewidth}{0.5mm}}
\DeclareMathOperator*{\argmin}{arg\,min}

\begin{document}
\begin{titlepage}
\pagenumbering{arabic}
\begin{center}
% \includegraphics[width=.7\textwidth]{images/uva}\\[1cm]

\HRule \\[0.4cm]
	{\huge \bfseries Human Computation \\in Online Video Storytelling}\\[0.4cm]
\HRule \\[1cm]%TODO check title + subtitle

\begin{tabular*}{0.95\textwidth}{@{\extracolsep{\fill}} l c r}
		&Philo van Kemenade&\\
		&philovankemenade [at] gmail [dot] com&	
\end{tabular*}\\[0.3cm]

\vspace{1cm}

Dissertation\\
MSc. Cognitive Computing\\

\vspace{0.5cm}

Goldsmiths College\\
University of London\\
% TODO Logo
\vspace{1.7cm}

\emph{Supervisor}\\
Dr. Marian Ursu
% TODO title

\vspace{0.25cm}
% Be sure to include (1) institute (if appropriate), (2) faculty (if
% appropriate), (3) organisation name, (4) organisation address (2 lines).
Department of Computing\\
Goldsmiths, University of London\\
New Cross\\
London SE14 6NW\\

\vfill 
\today

\end{center}
\end{titlepage}
\pagebreak


\begin{abstract}

\end{abstract}
\pagebreak 

\tableofcontents
\pagebreak

\section{Introduction}
\label{sec:introduction}

\section{The Quest for Meaningful Computation}
\label{sec:quest}


\subsection{The web as platform for creation}

Many media scholars have written about the role of the web.
Important trend of the web as platform of creation. In terms of video creation for example, the last few years have seen the development of online video editing tools and environments such as popcorn.js, WeVideo and Kaltura.

\section{Interactive Storytelling: From database to data-based}
\label{sec:storytelling}
\subsection{Symbolic approaches}
\label{sec:symbolic}
\subsection{Statistic approaches}
\label{sec:statistic}

\section{Never-ending Stories in `wePorter'}
\url{http://youtube-global.blogspot.co.uk/2009/05/zoinks-20-hours-of-video-uploaded-every_20.html}

\url{http://youtube-global.blogspot.co.uk/2010/03/oops-pow-surprise24-hours-of-video-all.html}

\url{http://youtube-global.blogspot.co.uk/2010/11/great-scott-over-35-hours-of-video.html}
\url{http://youtube-global.blogspot.co.uk/2011/05/thanks-youtube-community-for-two-big.html}
\url{http://youtube-global.blogspot.co.uk/2012/01/holy-nyans-60-hours-per-minute-and-4.html}

\section{Human Computed Stories in wePorter}
``Every second, an hour of video is uploaded to youtube''. This astonishing figure is nothing short of mind-boggling, but will most likely sound dated in a matter of years or even months. Looking at the increase of content uploaded to the video platform in past years, the growth does not seem likely to come to a halt soon. [include table] All these videos are great for online video junkies, and are increasingly part of the online journalism landscape. At the same time, all these videos being put online beg the question which ones of them to watch.

With more and more visual content being put online [ref], there is an increased need for ways of aggregation and filtering. Both of these tasks rely heavily on an understanding of what is presented in these media, which, as we've seen in the previous chapter, is a hard problem to solve via current computational techniques. With so much content being uploaded, how can we find our way in the already enormous ocean of online videos?

semantic annotation by tagging

more specific: once we have a repositiory that is returned as a search result because of annotations, how can we filter what is most interesting. (intentional ordering?)

\subsection{Current handles of meaning on uploaded video repositories}
tags etc


many people watching:
Clicking from one video to the next (choosing from a set of related videos)
these inter-video links could be seen as indicators for relatedness and relevance, much like google's page rank algorithm use links across webpages to establish a notion of the most significant site on a particular topic. 

There is an important difference here though. Whereas the links used by Google's search algorithms are embedded in machine readable hyperlinks, the path of clicking on from one video to the next is a characteristic of a person's interaction. 

differences:
	public, readable // private, non readable
	conscious choice // unconcious result of interaction
	Concluding
		can be consciously put in place by several people at large scale // dependent on real 'human' traffic.
	

\subsection{Analysis of the wePorter System}

\subsubsection{The death of the author}
ref: Barthes - Image music Sound
here: a lack of clear narrator of the written story

in wePorter: the lack of a centralised creator, authoring a work consciously and deliberately.

This new form of creation could at first sight be seen as non-authorship, but is rather a collective authorship. 

Several recent developments now make possible this collective form of authorship:
- the proliferation of tools hhat enable indicidual authorship of multimedia content
- the increased connectivity of these devices, which gives them the capacity to make the created media accessible to others
- the platforms for hosting multimedia (MM) content. This point is strongly related to the previous one. In ther development they form a chicken and egg relaten. Initially it's hard to imagine one withour the other, but once matured, they [influence eachother positively].
- the Methods to link, mix, aggregate and modify content online.


\label{sec:weporter}
\subsection{Implementation}
\label{sec:implementation}

$\underset{x}{\operatorname{argmin}}$

\begin{algorithm}
  \caption{My algorithm}
  \begin{algorithmic}[1]
    \Procedure{Euclid}{$a,b$}\Comment{The g.c.d. of a and b}
      \State $r\gets a\bmod b$
      \While{$r\not=0$}\Comment{We have the answer if r is 0}
        \State $a\gets b$
        \State $b\gets r$
        \State $r\gets a\bmod b$
      \EndWhile\label{euclidendwhile}
      \State \textbf{return} $b$\Comment{The gcd is b}
    \EndProcedure

  \end{algorithmic}
\end{algorithm}

% Pseudo Code generate Sequences Random Shuffled
% # fillSeqs
% (seq1, seq2) <- [],[]
% for i <- 0 to n_parts do
%   # select videoParts with different src than the parts already in sequence
%   selection1 <- []
%   selection2 <- []
%   for all vp in videoParts do:    
%     if vp.src not in selection1:
%       add vp to selection1
%     if vp.src not in selection2:
%       add vp to selection2            
% 
%   # seq1: select videoParts that have minimal count
%   minSelection11 <- []
%   minCount1 <- min(count) from selection1
%   for all vp in selection1:
%     if vp.count == minCount1:
%       add vp to minSelection1
%   selected1 <- pick random from minSelection1
%   add selected1 to seq1
% 
%   # seq2: filter videoParts with different src than selected for seq1
%   selection2V = []
%   for vp in selection2:
%     if vp.src != selected1.src:
%       add vp to selection2V
%   # select videoParts that have minimal count
%   minSelection2 <- []
%   minCount2 <- min(count) from selection2
%   for all vp in selection2:
%     if vp.count == minCount2:
%       add vp to minSelection2
%   selected2 <- pick random from minSelection2
%   add selected2 to seq2
% shuffle in unison(seq1, seq2)


\begin{algorithm}
  \caption{Generate Sequences Random Shuffled}
  \begin{algorithmic}[1]
    \Procedure{Fill Sequences}{}\Comment{The g.c.d. of a and b}
      \State $r\gets a\bmod b$
      \While{$r\not=0$}\Comment{We have the answer if r is 0}
        \State $a\gets b$
        \State $b\gets r$
        \State $r\gets a\bmod b$
      \EndWhile
      \State \textbf{return} $b$\Comment{The gcd is b}
    \EndProcedure

  \end{algorithmic}
\end{algorithm}



\section{Future Directions}
\label{sec:future}


\section{Conclusion}
\label{sec:conclusion}

\section{Acknowledgements} 
\label{sec:acknowledgements}

\bibliographystyle{plain} % different styles: plain, nature, jbact  
\bibliography{mybib}
   
\end{document}
