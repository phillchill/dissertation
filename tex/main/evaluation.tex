\chapter{Evaluation}
\label{ch:evaluation}

This is the Evaluation section.


\section{The Interface}
Positioning two video's one on top of the other, might inflict a bias for users in their attentional behaviour. It might be the case that videos on the top are systematically more attended to than videos displayed below. We've experimented to see whether such positioning bias effects occur and report on this in section \ref{sec:experiments} [*TODO ref].


\subsection{Landscapes of Interest}

The distinction to be made, between interesting intervals on one hand and less striking parts of a video on the other is not likely to be a very strict one. Afterall, an unedited video captures a single stretch of space and time, so any event that is of particular interest will unlikely have hard cut-off points in time. Rather, if we see interest as a function of time in a particular video, we would expect a somewhat continuous flowing line with spikes every now and then when an interesting event occurs. 

Looking at interest at a more global level, aggregating over a large group of users, would perhaps even a more smooth landscape of interest. This kind of data could reveal mountains and valleys that can be used for interest-based segmentation. From the thus segmented parts, the ones with high interest scores can be returned as the salient parts within a video.