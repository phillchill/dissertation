\chapter{Human Computation towards Visual Meaning}
\label{ch:humancomputation}

[not yet complete]

\section{Characterising Human Computation}
% [Define GWAP]

In a recent survey paper, Quinn and Bederson present a taxonomy of Human computation systems\cite{Quinn:2011us}. They sketch out the trend of this new method for intelligent problem solving by the increase of academic papers featuring the term `human computation' and its relative `crowd-sourcing'. They summarise the myriad of definitions given in recent works by several different authors in two key points:

\begin{itemize}
  \item ``The problems fit  the general paradigm of computation, and as such might someday be solvable by computers."
  \item ``The human participation is directed by the computational system of process"
\end{itemize}

The first point introduces an interesting question whether storytelling is a computable process.
In 1950, Alan Turing envisioned in his seminal paper `Computing Machinery and Intelligence' that a computer program would be able to successfully play a game now known as the Turing Test. His work also mentions that 

\begin{quote}
  ``[t]he idea behind digital computers may be explained by saying that these machines are intended to carry out any operations which could be done by a human computer''\cite{Turing:1950wi}
\end{quote}

The notion of `human computer' benefits from some contextualisation, as in the last few decades we've become unaccustomed to the term. Human computers were not uncommon in the time of Turing and before that from the 18th century, when `computer' was used to signify `one who computes'\cite{grier2007computers}. People bearing the function title were involved in the execution of calculations produced by strictly following mathematical theories. The activity that these computers were involved in was a process of rote, not requiring any human creativity. While working on the design for the first ever mechanical computer, Charles Babbage called it ``mental labour''\cite[Ch.\ 20]{Babbage:1832vu}.

Quinn and Bederson further present a classification along six dimensions they see as the most salient distinguishing factors: \cite{Quinn:2011us}

% [table of dimensions?]

\section{Humans Computing Visual Meaning}
% examples of HCS with the purpose of computing meaning
\subsection{Examples}
\subsubsection{ESP Game}
\subsubsection{Peek-a-Boom}
\subsubsection{reCaptcha}




\section{Computation in Interaction}


% many people watching collaborating
% is collective intelligence HC?
Clicking from one video to the next (choosing from a set of related videos)
these inter-video links could be seen as indicators for relatedness and relevance, much like google's page rank algorithm use links across webpages to establish a notion of the most significant site on a particular topic. 

There is an important difference here though. Whereas the links used by Google's search algorithms are embedded in machine readable hyperlinks, the path of clicking on from one video to the next is a characteristic of a person's interaction. 

% differences:
%   public, readable // private, non readable
%   conscious choice // unconcious result of interaction
%   Concluding
%     can be consciously put in place by several people at large scale // dependent on real `human' traffic.



\subsection{The web as platform for creation}
\label{sec:platform}
Many media scholars have written about the role of the web [refs New Media Reader].
Important trend of the web as platform of creation. In terms of video creation for example, the last few years have seen the development of online video editing tools and environments such as popcorn.js, WeVideo and Kaltura.



\section{Deriving Meaning from Video Via Human Factors}
[Personalized online document, image and video recommendation via commodity eye-tracking]\cite{Xu:2008vb}

[VideoReach: an online video recommendation system]\cite{Mei:2007wa}

\section{Collaborative Filtering} 

% \url{http://en.wikipedia.org/wiki/Collaborative_filtering}

The idea of using user's past interactions within a system hosting digital content for the filtering of items that might be of interest is not a new one and usually goes by the name of collaborative filtering. 
Collaborative filtering can generally take two forms: User-based, Item-based

Information filtering agents and collaborative filtering both
attempt to alleviate information overload by identifying
which items a user will find worthwhile.

% already happening at YouTube

\section{A Characterisation of Human Computation Systems}
\subsection{Purpose}
\subsection{Motivation}
\subsection{Task}

